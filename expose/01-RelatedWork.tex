\section{Related Work}
\subsection{Text Simplification}
According to Alva-Manchego et. al ``Text Simplification (TS) is the task of modifying the content and structure of a text in order to
make it easier to read and understand, while retaining its main idea and approximating its original meaning''.~\cite{AlvaManchego2020}
The simplified version of a text can assist e.g. non-native speakers, children or non-expert readers in their reading ability.~\cite{AlvaManchego2020, Jiang2020}

Text Simplification can be considered as a Sequence-to-Sequence Task, meaning a given input sequence of text can be mapped to an
output sequence of text, where the output sequence is a simplified version of the input.~\cite{Goldberg2017}
Therefore, some authors argue that TS is a monolingual translation task, where a complex input is translated into a simpler version.~\cite{Aumiller2022}
In addition to Machine Translation, Text Summarization is closely related to the Text Simplification task.
Most recent approaches to solve these Natural Language Processing (NLP) tasks are using Transformers, a deep-learning model based on an Encoder-Decoder architecture.
To train these models large amounts of data is necessary, for which reason pre-trained transformer models are commonly fine-tuned to a specific task.
%To train these models large parallel corpora are necessary.\cite{Jurafsky2023}\cite{AlvaManchego2020}
Fine-tuning is done using large parallel corpora. In the case of fine-tuning a text simplification model a large set of complex-simple sentence or document pairs is used,
depending on the granularity of the model to train.~\cite{Jurafsky2023} During training the Text Simplification model learns what modifications need to be made to transform a complex input into a simple one.
The transformations done by a trained sentence Text Simplification model ``range from replacing complex words or phrases for simpler synonyms, to changing the
syntactic structure of the sentence (e.g., splitting or reordering components)''.~\cite{AlvaManchego2020} This means that TS models can reduce 
lexical and syntactic complexity and explain complex concepts.~\cite{Saeuberli2020}

%TODO AUF ZEST PAPER eingehen!


%Text Simplification can be considered as a Sequence-to-Sequence Task, meaning the input of the model is a sequence of text and
%the output is a different sequence of text.\cite{Goldberg2017}


%Encoder-Decoder architecture
%Text Simplification is the task of modifying the content and structure of a text in order to make it easier 
%* What is TS?
%* What are Sequence-to-Sequence Tasks? Similiar to Machine Translation
%monolingual translation task (KLexikon)

%* Granularity of TS Approaches?
%    KLexikon Approach: Do Summarization and Simplification in one step
%* What requires TS? -> parallel corpora / alignments
%* Text Transformations/Operations of TS
%* How are Text Simplification Algorithms normally trained?

%Text Simplification hat zum Ziel, einen Text dahingehend zu verändern, dass er einfacher zu lesen und zu verstehen ist.
%The aim of Text Simplification is to modify a text so that it is easier to read and understand, while 
%The goal of Text Simplification (TS) is to modify a text so that it is easier to read and understand without changing the meaning.



\subsection{Adapters in the Transformer architecture}
Fine-tuned models depend on very large amounts of data and their performance increases with their number of parameters.~\cite{Pfeiffer2020}
This leads to models with billions of parameters, making it parameter inefficient.~\cite{Houlsby2019}
Therefore every fine-tuned model fits only one NLP task, and it is expensive to save and distribute these models.
This also makes composing different tasks more difficult.~\cite{Pfeiffer2020}
Adapters avoid this issues by placing layers in between the layers of a deep-neural net in a pretrained model.
The weights of the in between layers are updated during training phase, while the weights in the layers of the pretrained model stay unchanged.
Accordingly sharing the parameter weights is easier, e.g. to use them for another task.~\cite{Houlsby2019}
This makes Adapters suitable for transfer learning regard to different tasks or in cross-lingual contexts.
Adapterhub is a framework for building and integrating Adapters for Transformer-based language models and is based on
the Huggingface Transformers library.~\cite{Pfeiffer2023, Huggingface2023}

%Adapters are a in between layer within (feed-forward) neural networks
\subsection{Transfer Learning}
As previously mentioned fine-tuned models are tailored towards a specific Task and/or language, e.g. Text Simplification. 
% Can be applied to a different task and/or a different language